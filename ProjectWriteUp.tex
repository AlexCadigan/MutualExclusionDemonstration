% Document formatting:
\documentclass [11pt] {article}
\setlength {\parindent} {0pt}
\usepackage [margin = 1 in] {geometry}
\usepackage {hyperref}
\hypersetup {colorlinks=true, linkcolor=blue, filecolor=magenta, urlcolor=cyan}
\usepackage {graphicx}
\graphicspath { {img/} }
% Title:
\title {Mutual Exclusion Demonstration Write-Up}
\author {Jacob Naranjo, Tanush Samson, Alex Cadigan, Lionel Niyongabire}
% Builds document
\begin {document}
	\maketitle
	% Compiling and running
	\section {Compiling and Running}
	This program was tested on a laptop running MacOS.  These instructions for compiling and running the program are tailored for a MacOS environment.  These steps are not guaranteed to work on other systems.
	\begin {enumerate}
		\item Download \href {http://www.oracle.com/technetwork/java/javase/downloads/index.html} {Java}
  		\item \$ git clone https://github.com/AlexCadigan/MutualExclusionDemonstration.git
		\item \$ cd /MutualExclusionDemonstration/src/
		\item \$ javac MutualExclusionDemonstration.java Simulation.java Process.java
		\item \$ java MutualExclusionDemonstration
	\end {enumerate}
	% Description of program
	\section {Program Description}
	When the Mutual Exclusion Demonstration program is run, a GUI window will open asking for user input (Figure 1).\\  \includegraphics [scale = 1] {Figure1}\\\\  Figure 1 - Opening GUI window\\\\  In this first window, the user can select the speeds of the different objects.  The lower the speed, the faster the object will move.  When the user starts the simulation, a second GUI window will open, which will display the animation (Figure 2).\\\\  \includegraphics [scale = .5] {Figure2}\\  Figure 2 - Animation GUI window\\\\  The red path simulates the critical section of the distributed system, and the objects will first begin moving along the black paths (Figure 3).\\\\ \includegraphics [scale = .5] {Figure3}\\ Figure 3 - Objects moving along the path\\\\  When an object reaches the intersection of the critical section, it will use Lamport's algorithm to determine if it may enter the critical section (Figure 4).\\\\  \includegraphics [scale = .5] {Figure4}\\ Figure 4 - An object at the critical section intersection\\\\  Lamport's algorithm consists of 5 steps:
	\begin {enumerate}
		\item When an object comes to the critical section, it sends a time-stamped request to every other object in the system and also enters the request in its local queue.
		\item When an object receives a request, it places it in its queue.  If the receiving object is not in the critical section, it sends a time-stamped acknowledgement message back to the sender.  Otherwise, it defers sending the acknowledgement message until it exits from the critical section.
		\item An object may enter the critical section when (1) its request is ordered ahead of all other requests (i.e., the time stamp of its own request is less than the time stamps of all other requests) in its local queue and (2) it has received acknowledgement messages from every other object in response to its current request.
		\item To exit from the critical section, an object (1) deletes the request from its local queue and (2) sends a time-stamped release message to all the other objects.
		\item When an object receives a release message, it removes the corresponding request from its local queue.
	\end {enumerate}
	See figure 5 for an example of an object entering the critical section.  Note that each object must wait to enter the critical section until there are no objects in the critical section.\\\\ \includegraphics [scale = .5] {Figure5}\\ Figure 5 - An object moves through the critical section\\\\  The objects will continue moving until the user exits the GUI window.  All messages that are sent between objects are logged in a file for further analysis.  
	\section {Problems that Arose}
	\section {Overall Assessment}
\end {document}
